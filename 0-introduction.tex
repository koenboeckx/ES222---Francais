\chapter{Introduction}
\label{ch:introduction}
Ce cours est la suite du cours d'électronique donné par le Prof. dr. ir. Patrick MERKEN, qui a créé le contenu du cours et conçu la plupart des figures utilisées dans le travail. Mes remerciements vont à lui.\
\parindent=0pt

\section{Aperçu du cours}
Au début du cours, nous couvrirons les fondamentaux de la théorie des circuits, qui incluent les composants passifs tels que les résistances, les inductances et les capacités, ainsi que les lois de Kirchhoff, le comportement en fréquence des systèmes linéaires et la transformation de Thévenin et Norton. Nous supposons que les étudiants ont une compréhension de base de ces concepts.

La première grande section du cours, intitulée \emph{Composants}, introduira les étudiants aux bases de la physique des semi-conducteurs. Cela nous permettra de développer une compréhension des diagrammes de bande et des semi-conducteurs. Nous explorerons ensuite deux dispositifs à semi-conducteurs courants, à savoir les diodes et les transistors. Les deux types de transistors, à jonction bipolaire (BJTs) et à effet de champ à grille isolée en oxyde métallique (MOSFETs), seront couverts en détail.

Ensuite, nous passerons à la deuxième section du cours, intitulée \emph{Électronique analogique}. Ici, nous approfondirons le comportement des composants électroniques dans les circuits. Nous étudierons en détail les amplificateurs et les amplificateurs opérationnels (op-amps), ainsi que les oscillateurs, les références de tension et la théorie de la rétroaction. Un concept important que nous introduirons dans cette section est le modèle de petit signal, qui est utilisé pour étudier le comportement d'un circuit autour d'un point de fonctionnement.

Enfin, nous couvrirons la troisième section du cours, intitulée \emph{Électronique numérique}. Ici, nous explorerons comment les transistors peuvent être utilisés comme des interrupteurs pour construire des portes logiques pour effectuer des calculs. Nous discuterons également des circuits séquentiels qui contiennent des éléments de mémoire, ainsi que de la transition entre les domaines analogique et numérique. Cela comprendra une discussion sur les convertisseurs analogique-numérique et numérique-analogique.


\section{Une brève histoire de l'électronique}

L'électronique est un domaine qui a connu une évolution spectaculaire au cours du siècle dernier, avec ses origines ancrées dans les découvertes liées à l'électricité et au comportement des charges électriques. L'histoire de l'électronique est un voyage fascinant à travers les découvertes scientifiques, les innovations technologiques et l'évolution des dispositifs électroniques qui ont transformé notre façon de vivre et de travailler.

\subsection{Les premières découvertes et innovations}

Les racines de l'électronique remontent à la découverte de l'électricité par Benjamin Franklin au XVIIIe siècle. Cependant, ce n'est qu'au XIXe siècle que plusieurs découvertes et innovations clés ont préparé le terrain pour le développement de l'électronique moderne.

L'une des découvertes les plus importantes a été faite par Michael Faraday, qui a démontré au début des années 1800 qu'un champ magnétique changeant pouvait induire un courant électrique dans un fil à proximité. Ce phénomène, connu sous le nom d'induction électromagnétique, a ouvert la voie au développement de générateurs et de moteurs, qui sont devenus des composants essentiels de nombreux dispositifs électroniques.

En 1873, James Clerk Maxwell a publié un ensemble d'équations décrivant le comportement des champs électriques et magnétiques :
\begin{align*}
	\nabla \cdot \vec{E} &= \frac{\rho}{\epsilon} \\
	\nabla \cdot \vec{B} &= 0 \\
	\nabla \times \vec{E} &= -\frac{\partial \vec{B}}{\partial t} \\
	\nabla \times \vec{B} &= \mu \vec{J} + \epsilon \mu \frac{\partial \vec{E}}{\partial t}
\end{align*}
Ces équations ont unifié les théories de l'électricité et du magnétisme et ont prédit l'existence d'ondes électromagnétiques, qui se déplacent à la vitesse de la lumière. Cette découverte a été déterminante dans le développement de la communication radio, qui allait plus tard révolutionner notre manière de communiquer.

\subsection{Les premiers dispositifs électroniques}

Les premiers dispositifs électroniques ont émergé à la fin du XIXe siècle. En 1904, John Ambrose Fleming a inventé le tube à vide, qui était un type de tube électronique pouvant être utilisé comme amplificateur ou commutateur. Le tube à vide était un composant critique dans les premiers radios et télévisions et était la technologie principale utilisée pour l'amplification électronique jusqu'au développement du transistor au milieu du XXe siècle.

En 1907, Lee De Forest a inventé le triode, un type de tube à vide capable d'amplifier des signaux électriques en contrôlant le flux d'électrons à travers le vide. Le triode a rendu possible l'amplification de signaux avec une grande précision, en faisant un composant essentiel des premiers radios et autres dispositifs électroniques.

\subsection{L'avènement de l'électronique moderne}

Le développement du transistor en 1947 a marqué un tournant significatif dans l'histoire de l'électronique. Le transistor, inventé par John Bardeen, Walter Brattain et William Shockley chez Bell Labs, était un dispositif à semi-conducteurs qui pouvait être utilisé comme commutateur ou amplificateur. Les transistors étaient plus petits, plus rapides et plus fiables que les tubes électroniques et ont permis la miniaturisation des dispositifs électroniques.

Le développement du circuit intégré dans les années 1950 et 1960 a encore révolutionné l'électronique en permettant à plusieurs transistors et autres composants d'être fabriqués sur une seule pièce de silicium. Cela a permis la création de systèmes électroniques complexes qui pouvaient être beaucoup plus petits, plus légers et plus abordables que jamais auparavant.

Au cours des dernières décennies, l'électronique a continué à évoluer avec le développement de nouvelles technologies telles que les microprocesseurs, le traitement numérique du signal et la communication sans fil. Aujourd'hui, l'électronique joue un rôle essentiel dans presque tous les aspects de la vie moderne, des smartphones et des ordinateurs aux dispositifs médicaux et aux technologies d'énergie renouvelable.
